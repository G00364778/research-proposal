% LaTeX generates professionally typeset documents. It's different
% from word processors like Microsoft Word in that it separates style from
% content. It lets you write content without worrying too much about things like
% fonts, positioning of tables, and references. LaTeX has carefully considered
% rules to that sort of that heavy-lifting.

% LaTeX files are text files where most of the content is the words, sentences,
% and paragraphs. When complete, LaTeX files are fed into the pdflatex program
% to create a pdf - we usually call this compiling the document.

% Within the text you can put comments and commands. These lines are comments.
% On any line, everything after a percent sign (%) is treated as a comment.
% Comments don't have any effect on the output file.

% Commands start with a backslash. The first command is usually \documentclass,
% as below. It tells LaTeX the type of document to generate: book, article,
% letter, etc. Commands can take parameters and/or options.

% Parameters appear in curly braces after the command. Options appear in square
% brackets between the command and the parameters. The \documentclass command
% below has "article" as a parameter and "a4paper" as one option. Multiple
% options are separated by commas.

% Some commands come in pairs like \begin{center} and \end{center}, which is
% usually called an environment. Content in a center environment is centered.

% LaTeX is built on a language called TeX. TeX is a simple language with only a
% few built-in commands. TeX allows us to create our own commands. LaTeX is a
% huge collection of such programmer-created commands in one handy package.

% There are only a few types of document built-in to LaTeX, like article.
% If you google "latex article options" you'll find lots of sites listing
% what each option does.
\documentclass[a4paper, oneside, hidelinks]{article}

% Lots of LaTeX programmers have created extra packages for LaTeX.
% They're not enabled by default as they affect the length of time it takes
% to compile the document. They can enable things like colour and hyperlinks
% in the output pdf. The \usepackage command will include a package, so long
% as it is installed along with LaTeX. TeXLive include lots of standard
% packages by default. Google "latex minted" to see what it does.

% Enables the use of colour. 
\usepackage{xcolor}
% Syntax high-lighting for code. Requires Python's pygments.
\usepackage{minted}
% Enables the use of umlauts and other accents.
\usepackage[utf8]{inputenc}
% Diagrams.
\usepackage{tikz}
% Settings for captions, such as sideways captions.
\usepackage{caption}
% Symbols for units, like degrees and ohms.
\usepackage{gensymb}
% Latin modern fonts - better looking than the defaults.
\usepackage{lmodern}
% Allows for columns spanning multiple rows in tables.
\usepackage{multirow}
% Better looking tables, including nicer borders.
\usepackage{booktabs}
% More math symbols.
\usepackage{amssymb}
% More math fonts, like mathbb.
\usepackage{amsfonts}
% More math layouts, equation arrays, etc.
\usepackage{amsmath}
% More theorem environments.
\usepackage{amsthm}
% More column formats for tables.
\usepackage{array}
% Adjust the sizes of box environments.
\usepackage{adjustbox}
% Better looking single quotes in verbatim and minted environments.
\usepackage{upquote}
% Better blank space decisions.
\usepackage{xspace}
% Better looking tikz trees.
\usepackage{forest}
% URLs.
\usepackage{hyperref}
% Plotting.
\usepackage{pgfplots}
% Filler text.
\usepackage{lipsum}
% Line spacing.
\usepackage{setspace}
% Changing spacing on pages.
\usepackage{geometry}
% Gantt charts.
\usepackage{pgfgantt}
% Better appendices.
\usepackage{appendix}

% Use latest pgfplots.
\pgfplotsset{compat=newest}

% The tikz package allows us to create diagrams and plots using LaTeX
% commands. It defines some extra commands, like \usetikzlibrary below.

% Various tikz libraries.
% For drawing mind maps.
\usetikzlibrary{mindmap}
% For adding shadows.
\usetikzlibrary{shadows}
% Extra arrows tips.
\usetikzlibrary{arrows.meta}
% Old arrows.
\usetikzlibrary{arrows}
% Automata.
\usetikzlibrary{automata}
% For more positioning options.
\usetikzlibrary{positioning}
% Creating chains of nodes on a line.
\usetikzlibrary{chains}
% Fitting node to contain set of coordinates.
\usetikzlibrary{fit}
% Extra shapes for drawing.
\usetikzlibrary{shapes}
% For markings on paths.
\usetikzlibrary{decorations.markings}
% For advanced calculations.
\usetikzlibrary{calc}

% No prizes for guessing what the following few commands do.

% GMIT colours.
\definecolor{gmitblue}{RGB}{20,134,225}
\definecolor{gmitred}{RGB}{220,20,60}
\definecolor{gmitgrey}{RGB}{67,67,67}

% Uncomment for one and a half line spacing.
% \onehalfspacing

% Tell minted to use the following colour scheme. 
\usemintedstyle{manni}
% Set some minted options.
\setminted{frame=lines, framesep=2mm, baselinestretch=1.2, linenos}

% The title.
\title{Research proposal: \\ Multi-zone Thermal PID modelling using Neural networks}
\author{Gerhard van der Linde (G00364778@gmit.ie)}
\date{\today}

% Everything above the below \begin{document} command is referred to as the
% preamble. The preamble configures the features and styles of the output.
% The \begin{document} command tells LaTeX we beginning the content of the
% document - the words, sentences, and paragraphs we want to create.

% Begin the document.  
\begin{document}
  % Display the title in the standard way for the document class.
  \maketitle
  \begin{center}
  \includegraphics[width=200px]{img/gmit-logo.jpg}
  \end{center}

  % Usually, the first item after the title is an abstract.
  % An abstract gives a short summary of the main contribution of the document.
  \begin{abstract}
    We propose to research and develop and improvement a three-zone neural
    network to model and predict the PID parameters of temperature controller. 
    The model will adjust the various PID terms based on the variances in the 
    process conditions like target temperature, thermal ramp rate, thermal load, 
    chamber pressure, process gas type and other external factors with a 
    potential influence on the chamber thermal performance.

    The purpose of the modelling is to provide a single solution capable of 
    achieving multiple target temperatures with a tight distribution of 
    temperatures across a stack of silicon wafers and across the wafer 
    surfaces withing the target deviation specification during the three stages 
    of the thermal cycle, i.e. heating, steady state and cool down.   
    
    The software program will create a model from existing thermal profiles and 
    recorded process variables and attempt to construct a time based neural network 
    model that can predict and transmit the PID terms to use given a predetermined 
    set of process requirements and a real time measured set of environment variables
    with a potential impact on the process performance.

  \end{abstract}

  % We generally organise our document then in sections. A section is the
  % highest level of organisation, then subsection, then subsubsection. The
  % second subsection of the third section would be listed as part 3.2 of the
  % document. You can either use the command \section{My section} to begin a 
  % new section or you can use \begin{section}{My section} and \end{section} to
  % the same effect. The former is generally preferred as it's cleaner and LaTeX
  % can generally figure out where the section ends.
  \section{Introduction}
    The fundamental temperature challenge faced is to achieve a very tight thermal
    gradient over a stack of semiconductor silicon wafers, across the stack and 
    across the individual wafer surfaces of plus/minus 1.5 \degree C during ramping 
    up and steady state to a target temperature of anything from around 200 \degree 
    C to 600 \degree C.

    The challenge is compounded by the space constrains driven by the superconducting 
    magnet wrapped around the thermal process. Increasing the bore size of the oven 
    to facilitate a batch oven running a thermal process inside a magnet, exponentially 
    increase the size, weight and cost of the superconducting magnet system.
  
    The optimal thermal solution is to run a multi-zone element with multi zone PID 
    controls to balance the thermal load on the wafer stack, however other external 
    factors influences the thermal behavior and interaction of zones.
  
    The three zone controller approach also creates PID challenges as thermal zones 
    interact, so PID tuning is quote tedious and time consuming. Tuning multiple 
    target temperatures for a system can take weeks to hit the specifications for all 
    target temperature simultaneously.
  
    The challenge is compounded further by other external factors that also have a 
    significant influence on temperature performance once tuned.

    \begin{itemize}
      \item Multiple target temperatures between 200-600 \degree C per system.
      \item Variance in thermal load of product wafers varying by wafer types and 
      deposition and variance in batches. 
      \item Variance in thermal emissivity of different products
      \item Variance in process chamber condition running in vacuum or various process 
      gasses like Helium, Nitrogen, Argon etc. 
      \item Other external factors that have potential influence on the processes, 
      like temperatures of cooling water loops, room temperatures, atmospheric conditions 
      like moisture, starting temperature of thermal job, for example the time duration 
      started after previous job.
    \end{itemize}

    
    The introduction should also give a short overview of the rest of the
    document -- a one or two sentence overview of each part of the document.

  % Second section.
  \section{Literature}
    The idea of using modelling to solve the long standing PID problem started with 
    the first machine learning projects to demonstrate the functionality of neural networks.
    
    The concept was further encouraged by the article describing a similar PID problem 
    solved for tuning a variety of PID loops for underwater vehicles with constantly 
    changing ballast and ocean conditions.~\cite{ArticleUnderwatervehicle}

    Conceptually the idea for the theses is that given access to around 20 years 
    worth of real time  thermal process information across a fleet of thermal 
    processing tools, there should be enough data 
    to build a neural network using all the variables listed previously and using 
    existing control parameters with the potential addition of other potential 
    unrecorded variables to improve the predictions, one should be able to create a 
    neural network to model the three decoupled thermal PID control zones and underlying 
    PID terms to compensate for variations dynamically as they occur.
   
    While this sound simple and straight forward, the challenging aspects for this
    approach would be the big time lags in the PID control system and finding time based 
    modelling methods to facilitate the solution. From doing some basic initial 
    checks for literature and tools it seems feasible to achieve.
    ~\cite{InprocDecoupledPIDNeeuralNet}

    Although the application is slightly different the decoupling concept of PID 
    zones is very similar and the concepts should translate. 
    
    To elaborate on the thermal challenge~\cite{LargeWaferThermalBahaviour}, thinking 
    about a stack of twenty five, 200mm or 300mm silicon wafers, spaced a few 
    millimeters apart, pretty much 
    looks like a solid mass to a thermal process and there is a direct correlation 
    of the thermal gradient to the speed of thermal increase, in other words the 
    speed at with the target set-point is driven.~\cite{WaferThermalUniformity} 
    The thermal gradient is reduced by 
    introducing a ramp rate on the target temperature set-point and is typically achieved 
    over the course of an hour or more, so modelling techniques will have to consider 
    time delay compensation and this will be the challenging part of the project as 
    far as I can determine.~\cite{InprocDecoupledPIDNeeuralNet}
 

    % It's important to refer to the existing literature in your proposal. In a
    % research proposal I would expect to see at least three or four references
    % directly related to the proposal. The next sentence includes a reference.
    % References can be inserted using the \cite{} command with the item's
    % bibtex name in the curly braces. This will insert the reference indicator
    % where the \cite command is called and also ensure the reference title,
    % authors, etc in the list of references. The ~ controls the amount of
    % whitespace around the reference, try Googling "latex reference tilde".
    % We can give most high-level items a label with the \label command.
    
    % We'll come back to that in a minute.
    \label{section:literature}
    % Typically, a research proposal will give a bit of background, listing four
    % or five directly relevant peer-reviewed publications.

  % Another section.
  \section{Research question}
    % All research starts out with an idea. There is no shortage of interesting 
    % ideas. An idea needs to be developed into a research question to have a
    % reasonable chance of leading to a successful thesis submission.
    
    % A research question has a few essential characteristics. It should be clear 
    % and specific. It should be informed by the research work of others, as
    % described in peer reviewed publications, books, and other reliable sources.
    % It should lend itself to an investigation of some sort, that either provides
    % evidence as to its answer or answers it outright. The investigation needs to
    % be reasonable to complete in the time frame of the proposed research project.

    % It can be difficult to formulate a research question that meets all those
    % characteristics. Research by its nature is tentative. It may seem impossible
    % to know ahead of time whether a given investigation could be completed in a
    % given time frame.
    
    % The purpose of a research proposal, however, is to demonstrate an
    % understanding of the what will be involved in carrying out your research
    % project. You will, in consultation with your supervisors, be able to adapt
    % your research topic, research question, deliverables, and milestones as the
    % project evolves.

    % Everything is geared towards helping you to succeed in submitting a
    % successful thesis.

    The basic idea is to design and train a neural network to dynamically model
    and predict appropriate PID sets for the variances in process conditions and 
    adjust decoupling and the degree of interaction between the overlapping thermal 
    control zones.

    \begin{itemize}
      \item Can we establish meaningful correlations between inputs and outputs
      \item Can we find a time delayed solution for parameter predictions
      \item Can we measure and sufficiently decouple control zones and determine
            decoupling constants
      \item Can we subsequently find a unified model that accurately predicts the PID sets
      \item Can we get the model to feed the predictions dynamically to our PID controller
    \end{itemize}

  % Another section.
  \section{Timeline}
    It is estimated that this project will take two years to complete. The Gantt 
    chart in Figure \ref{figure:ganttchart} gives an overview of the expected timeline.

    
    \begin{figure}[H]
      \begin{center}
        \begin{ganttchart}[title/.style={draw=none},
                           vgrid, hgrid,
                           canvas/.append style={draw=gmitgrey},
                           bar/.append style={fill=gmitgrey!60}]{1}{24}
          \gantttitle{Q4'20}{3}
          \gantttitle{Q1'21}{3}
          \gantttitle{Q2'21}{3}
          \gantttitle{Q3'21}{3}
          \gantttitle{Q4'21}{3}
          \gantttitle{Q1'22}{3}
          \gantttitle{Q2'22}{3}
          \gantttitle{Q3'22}{3} \\
          \ganttbar{Literature}{1}{3} \\
          \ganttbar{Develop}{4}{12} \\
          \ganttbar{Test}{13}{15} \\
          \ganttbar{Finalise}{16}{24}
        \end{ganttchart}
      \end{center}
      \caption{Gantt Chart}
      \label{figure:ganttchart}
    \end{figure}


  % Let's create a subsection.
  \subsection{Milestones}
    The project will be delivered through a series of milestones. The milestones
    will have the follow deliverables.
    % Here's a description list. Bulleted and numbered lists are similar.
    \begin{description}
      \item[Literature review:] this will be delivered at the end of month 3.
      \item[Sofware Model:] this will be delivered at the end of month 12.
      \item[Tests:] the tests will be completed at the end of month 15.
      \item[Thesis:] the thesis will be delivered at the end of month 24.
    \end{description}


  % Now that we have all of the sections for a research proposal done, we'll
  % add an appendix just to show off what LaTeX can do.
  
  % First, move to a new page.
  % \newpage
  % % Next tell LaTeX to start the appendices.
  % \appendixpage
  % \appendix

  % % Now sections will be given letter identifiers rather than numbers.
  % \section{Math, images, code, and tables}
  %   % Everything else remains the same.
  %   \LaTeX{} contains many features for incorporating various visual elements.
    
  %   \subsection{Math}
  %     For instance, \LaTeX{} is great at rendering math.
  %     $$ f:\mathbb{R} \rightarrow \mathbb{R}:x \rightarrow e^{x+1} $$


  %   \subsection{Plots}
  %     You can create plots programmatically from within \LaTeX{} like in Figure 
  %     \ref{plot:xsqfour}.
  %     \begin{figure}[H]
  %       \begin{center}
  %         \begin{tikzpicture}
  %           \begin{axis}[
  %             xlabel=$x$,
  %             ylabel={$f(x) = x^2 - x +4$}
  %           ]
  %           % use TeX as calculator:
  %           \addplot {x^2 - x +4};
  %           \end{axis}
  %         \end{tikzpicture}
  %       \end{center}
  %       \caption{Great at plots.}
  %       \label{plot:xsqfour}
  %     \end{figure}

  %   \subsection{Images}
  %     Latex can incorporate raster images like PNG's and JPG's, as in Figure
  %     \ref{img:logo}.
  %     \begin{figure}[H]
  %       \begin{center}
  %         \includegraphics[width=200px]{img/gmit-logo.jpg}
  %       \end{center}
  %       \caption{The GMIT logo.}
  %       \label{img:logo}
  %     \end{figure}
    
  %   \subsection{Mind maps}
  %     The TikZ package is the main plotting library for \LaTeX{}.
  %     It can create mind maps like in Figure \ref{mindmap}.

  %     \begin{figure}[H]
  %       \begin{center}
  %         \resizebox{0.8\textwidth}{!}{%
  %           \begin{tikzpicture}
  %             \path[mindmap,concept color=gmitgrey,text=white]
  %               node[concept] {Computable}
  %               [clockwise from=0]
  %               child[concept color=gmitblue] {
  %                 node[concept] {Recursive}
  %                 [clockwise from=30]
  %                 child {node[concept] {Decidable}}
  %                 child {node[concept] {Not}}
  %               }
  %               child[concept color=gmitred] {node[concept] {Turing}}
  %               child[concept color=gmitblue] {node[concept] {Reduction}}
  %               child[concept color=gmitred] {
  %                 node[concept] {Complexity}
  %                 [clockwise from=210]
  %                 child { node[concept] {P} }
  %                 child { node[concept] {NP} }
  %               };
  %           \end{tikzpicture}
  %         }
  %       \end{center}
  %       \caption{A mind map.}
  %       \label{mindmap}
  %     \end{figure}

  %   \subsection{Automata}
  %     It can also create automata like in Figure \ref{automaton}.
  %     \begin{figure}[H]
  %       \begin{center}
  %         \begin{tikzpicture}[auto, on grid, node distance=2cm, initial text=, >=latex]
  %           \node[state, initial]   (q_1)                {$q_1$}; 
  %           \node[state]            (q_2) [right of=q_1] {$q_2$};
  %           \node[state]            (q_3) [right of=q_2] {$q_3$}; 
  %           \node[state, accepting] (q_4) [right of=q_3] {$q_4$};
  %           \path[->] 
  %             (q_1) edge [loop above] node {$0,1$}        ()
  %                   edge []           node {$1$}          (q_2)
  %             (q_2) edge []           node {$0,\epsilon$} (q_3)
  %             (q_3) edge []           node {$1$}          (q_4)
  %             (q_4) edge [loop above] node {$0,1$}        ();
  %         \end{tikzpicture}
  %       \end{center}
  %       \caption{An automaton.}
  %       \label{automaton}
  %     \end{figure}



  % LaTeX together with bibtex will generate a list of references from the
  % \cite commands in the LaTeX document and the items in the bib file.

  % The style of referencing is controlled by the \bibliographystyle command.
  % There are loads of styles available - google "bibtex styles".
  % The ieeetr style uses numbers for references which is common in computing.
  \bibliographystyle{ieeetr}
  % The \bibliography command tells LaTeX where the bib file is. The bib file
  % is called "bibliography.bib" in the current folder. You omit the ".bib".
  \bibliography{bibliography}

% The end of the document.
\end{document}
